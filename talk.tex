\documentclass[aspectratio=169]{beamer}
\usetheme{Pittsburgh}
\usecolortheme{spruce}
\usefonttheme{structurebold}
\beamertemplatenavigationsymbolsempty
\usepackage{graphicx}
\usepackage{tikz}

\title{Far Flung Forest Landscapes in the Anthropocene}
\subtitle{Structural analysis of China's embodied forest network}
\author{M.K. Lau (Ph.D.)}
\institute{Chinese Academy of Sciences and Harvard University}


\begin{document}

\begin{frame}
  \titlepage
\end{frame}


\begin{frame}
  \frametitle{Overview}

\tableofcontents

%% - Intro/Context
%%   - Forests are globally important
%%   - Anthropocence effects 
%%   - Global forest loss and gain and change
%% 	- Global greening = India(Agriculture) + China(Forests)
%% - Economics*Ecology = Landscape Extended Models
%% - Network Analysis of China's Greening
%%   - Global Scale
%%   - Local Scale
%% 	- Landscape = Chen 2019
%% 	- Resilience Analysis of China's Forest LE-MRIO
%% - Conclusions and Future Work
%% - Acknowledgements

\end{frame}


\section{Context}

\begin{frame}
  \frametitle{Forests are Important Globally}

  \begin{itemize}
  \item biodiversity
  \item water and nutrient cycling
  \item carbon storage
  \item resources(wood, food)
  \item culturally
  \end{itemize}

\end{frame}




\begin{frame}{Q \& A}

\end{frame}

\end{document}
